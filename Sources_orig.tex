\section{Sources}\label{sec:sources}
All the sources described here are plotted drawn as boxes in figures \ref{fig:hc}, \ref{fig:S} and \ref{fig:omega}. The boxes are drawn in such a way that there is a reasonable event rate for sources at a detectable signal to noise ratio. However both of these criteria are somewhat vague and detector specific so we describe the exact box we plot for each source in the corresponding section. We draw sources with short duration (i.e. burst sources) and sources which evolve in time over much longer timescales than our observations with flat topped boxes. Inspiralling sources which change there frequency over observable timescales are drawn with slopped tops. The slope chosen for the top of the box is a gradient of $-2/3$ which is also the gradient for the stochastic backgrounds of binaries. 



\subsection{Sources for ground based detectors}
A mass has an associated gravitational timescale $\tau=\frac{GM}{c^{3}}$ which is, up to a geometric factor, the light crossing time of a black hole of mass $M$. The sources described in this section all have a mass ${\cal{O}}(1-10)\Msun$, which gives a characteristic frequency range of ${\cal{O}}(10^{4}-10^{5})\textrm{Hz}$. All the sources described in this section actually radiate in the lower frequency range ${\cal{O}}(10-10^{3})\textrm{Hz}$ this is because they are all more spatially extended than a single black hole and hence the light crossing time is longer. Another potential source of gravitational waves in this frequency range not discussed here is rotating, non-spherical, neutron stars, see \cite{2013ASPC..467...59S}.

\subsubsection{Supernova}
Simulations of gravitational waves from a core collapse supernova event produce radiation from approximately $(10^{2}-10^{3})\,\textrm{Hz}$, \cite{2002A&A...393..523D}. Because the signal is burst like, and only undergoes ${\cal{O}}(1)$ oscillation the signal doesn't accumulate signal to noise in the same way as an inspiral event does, so we use the vociferosity relation in (\ref{eq:simple}). \cite{2002A&A...393..523D} calculate the average maximum amplitude of gravitational waves for a supernova at distance $r$ as, 
\begin{equation} h_{0}=8.9\times 10^{-21}\left( \frac{10 \,\textrm{kpc}}{r} \right) \quad .\end{equation}
Adopting an expected event rate for supernova, e.g. \cite{2013ASPC..467...59S}.
\begin{equation} \rho_{\textrm{SN}}=5\times10^{-4}\textrm{Mpc}^{-3}\textrm{yr}^{-1} \end{equation}
Choosing a distance, $r\approx 3 \,\textrm{Mpc}$, such that we expect one supernova per 10 years, gives an amplitude of $h_{c}=10^{-22}$. Hence what is plotted in figures \ref{fig:hc}, \ref{fig:S} and \ref{fig:omega} is a box between $10^{2}\,\textrm{Hz}$ and $10^{3}\,\textrm{Hz}$, with a height corresponding to $h_{c}=10^{-22}$.

\subsubsection{Neutron star binaries}
The inspiral and merger of a pair of neutron stars is the prime candidate source for ground based detectors. The last few orbits and merger, which produces the largest amplitude waves, occurs over ${\cal{O}}(10)$ orbits, so as with the supernova the vociferosity relation in (\ref{eq:simple}) is used. The typical strain generated by an event that released $E_{\textrm{GW}}$ at a distance $r$, centred around frequncy $f$ and with a duration $\tau$ is given by \cite{2013ASPC..467...59S}.
\begin{equation} h_{0}\approx 5\times 10^{-21}\left(\frac{E_{\textrm{GW}}}{10^{-7}\Msun c^{2}}\right)^{1/2}\left( \frac{\tau}{1\,\textrm{ms}} \right)^{-1/2}\left( \frac{f}{1\,\textrm{kHz}} \right)^{-1}\left( \frac{r}{10\,\textrm{kpc}} \right)^{-1} \end{equation}
The expected event rate for this type of event is uncertain, but estimates centre around (see, for example, \cite{2011PrPNP..66..239A}),
\begin{equation} \rho_{\textrm{NS-NS}}=10^{-6}\;\textrm{Mpc}^{-3}\textrm{yr}^{-1} \quad .\end{equation}
Plotted in figures \ref{fig:hc}, \ref{fig:S} and \ref{fig:omega} is a box with an amplitude equal to an event releasing $10^{-4}\Msun c^{2}$ of energy, centred at a frequency of $100\,\textrm{Hz}$, over a period of $10\,\textrm{ms}$ and at a distance of $10\,\textrm{Mpc}$.










\subsection{Sources for space based detectors}
Space based detectors are sensitive to lower frequency gravitational waves than their ground based counterparts, typically $10^{-4}\,\textrm{Hz}-10^{-2}\,\textrm{Hz}$. This is partly because space based detectors can have much longer arms and partly because they are unaffected by seismic noise which limits the low frequency performance of ground based detectors. The loudest predicted source in this frequency range is the merger of supermassive black holes associated with galaxy mergers \cite{JohnsLivingReview}. The classic space based detector is LISA, all the sources discussed here lie within LISA's sensitivity curve.

\subsubsection{Massive black hole coalescences}
The majority of galaxies contain a supermassive BH in their centres, space based detectors will be sensitive to equal mass mergers in the mass range $10^{4}-10^{7}\,\Msun$ \cite{JohnsLivingReview}. Predictions of the event rate for these megers range from ${\cal{O}}(1-100)\,\textrm{yr}^{-1}$ for LISA with signal to noise ratios of up to 1000. The uncertainty in this rate reflects our uncertainty in the growth mechanisms of the supermassive black hole population. Plotted in figures \ref{fig:hc}, \ref{fig:S} and \ref{fig:omega} is a box with signal to noise ratio of 100 for eLISA at it's peak sensitivity. The range of frequencies plotted is $3\times 10^{-4}\textrm{Hz}$ and $3\times 10^{-1}\textrm{Hz}$, this corresponds to circular binaries in the mass range quoted above.

\subsubsection{Galactic binaries}
These divide into two classes, the unresolvable galactic binaries and the resolvable galactic binaries which lie at higher frequencies and amplitudes. The distinction between resolvable and unresolvable is detector specific, here we choose LISA. This boundary will not be too different for eLISA but would move substantially for either of the decihertz detectors, BBO or DECIGO. For the \emph{unresolvable} binaries the box is plotted in the knee of the LISA noise curve, this unfortunately makes them look \emph{undetectable}, however the entire population will be detectable but only as a stochastic background.

For the resolvable binaries the event rates range from ${\cal{O}}(10^{4})$ for LISA to ${\cal{O}}(10^{3})$ for eLISA, see \cite{JohnsLivingReview}. Signal to noise ratios for these events observed with eLISA will extend to above 50, see \cite{DoingScienceWitheLISA}. The box plotted in figures \ref{fig:hc}, \ref{fig:S} and \ref{fig:omega} has an signal to noise ratio of 50 for eLISA at it's peak sensitivity, the frequency range of the box is based on visual inspection of Monte carlo population simulation results presented in \cite{DoingScienceWitheLISA}.

\subsubsection{Extreme mass ratio inspirals}
Extreme mass ratio inspiral (EMRI) events occur when a compact stellar mass object, i.e. a black hole or neutron star, inspirals into a supermassive black hole in the centre of a galaxy. There is extreme uncertainty in the event rate for EMRIs due to the poorly constrained astrophysics in galactic centres; event rates estimates per galaxy range from ${\cal{O}}(1-10^{3})\,\textrm{Gyr}^{-1}$ for black hole EMRIs and ${\cal{O}}(10-5000)\,\textrm{Gyr}^{-1}$ for white dwarfs and neutron star EMRIs. The box plotted in figures \ref{fig:hc}, \ref{fig:S} and \ref{fig:omega} has a characteristic strain of $h_{c}=3\times 10^{-20}$ at $10^{-2}\;\textrm{Hz}$ which corresponds to a $10\Msun$ BH inspiralling into a $10^{6}\Msun$ supermassive black hole with a last stable orbital eccentricity of $e=0.3$ at a distance of $1\,\textrm{Gpc}$. The frequency width of the box is somewhat unknown, EMRI events can occur into a black hole of any mass, and hence EMRIs can occur at any frequency. However there is substantial uncertainty in the population statistics of supermassive black holes, particularly of intermediate mass, $(10^{2}-10^{4})\,\Msun$.












\subsection{Sources for pulsar timing arrays}
Pulsar timing arrays (PTAs) can be thought of a naturally occuring interferometers with galactic scale arms, hence PTAs are sensitive to much lower frequency gravitational waves the the detectors considered so far. There are several extremely interesting potential cosmological sources of gravitational waves in this frequency band, for example cosmic strings networks and the reheating process, see for example \cite{Haasteren}. However, here we ristrict ourselves to sources which are somewhat more certain, the redshifted mergers of ${\cal{O}}(10^{7}-10^{10})\Msun$ black holes associated with galaxy mergers at cosmological distances.

\subsubsection{Stochastic background of supermassive black hole inspirals}
The current best published limit for the amplitude of the stochastic background is $h_{c}=6\times 10^{-15}$ at frequency of $\textrm{yr}^{-1}$, see \cite{Haasteren}. There is strong theoretical evidence that the actual background lies close to the current limit, see \cite{imminentdetectionofgravitationalwaves} and \cite{NONimminentdetectionofgravitationalwaves}. As the frequency increases, the sources become less redshifted and hence closer and louder. Hence at a certain frequency they will cease to be a background and become individually resolvable. The exact point at which a source become individually resolvable is not exactly defined, and will also depend upon the detector used; plotted in figures \ref{fig:hc}, \ref{fig:S} and \ref{fig:omega} is a third of the limit due to \cite{Haasteren} with a cut off frequency of $\textrm{yr}^{-1}$ which is suggested by Monte-Carlo population studies, see \cite{SesanaVecchioColancino}.

\subsubsection{$10^{9}$ solar mass binaries}
If an inspiral between two supermassive black holes is close and loud enough it will become individually resolvable. These sources would be expected to occur at higher frequencies (i.e. closer and less redshifted) than the stochastic background. Plotted in figures \ref{fig:hc}, \ref{fig:S} and \ref{fig:omega} is the current EPTA limit, \cite{Haasteren}.



