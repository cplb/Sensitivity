\section{Astrophysical sources}\label{sec:sources}
All the sources described here are drawn as boxes in figures \ref{fig:hc}, \ref{fig:S} and \ref{fig:omega}. The boxes are drawn in such a way that there is a reasonable event rate for sources at a detectable signal-to-noise ratio. The exact description for each source is given in the corresponding section. We draw sources with short duration (i.e. burst sources) and sources which evolve in time over much longer timescales than our observations with flat topped boxes. Inspiralling sources which change their frequency over observable timescales are drawn with sloping tops. The slope for the top of the box is arbitrarily chosen as $-2/3$ which is also the exact gradient for the stochastic backgrounds of binaries. 




\subsection{Sources for ground based detectors}
Another potential source of GWs in this frequency range not discussed further here is rotating, non-spherical, neutron stars, see \cite{2013ASPC..467...59S}.

\subsubsection{Neutron star binaries}
The inspiral and merger of a pair of neutron stars is the primary target for ground based detectors. The detectable portion of the inspiral occurs over the last ${\cal{O}}(10)$ orbits, so to a reasonable aproximation it can be treated as a burst source. The typical strain generated by an event that released $E_{\textrm{GW}}$ at a distance $r$, centred around frequency $f$ and with a duration $\tau$ is given by \cite{2013ASPC..467...59S}.
\begin{equation} h_{0}\approx 5\times 10^{-21}\left(\frac{E_{\textrm{GW}}}{10^{-7}\Msun c^{2}}\right)^{1/2}\left( \frac{\tau}{1\,\textrm{ms}} \right)^{-1/2}\left( \frac{f}{1\,\textrm{kHz}} \right)^{-1}\left( \frac{r}{10\,\textrm{kpc}} \right)^{-1} \end{equation}
The expected event rate for this type of event is uncertain, but estimates centre around $\rho_{\textrm{NS-NS}}=10^{-6}\;\textrm{Mpc}^{-3}\textrm{yr}^{-1}$, \cite{2011PrPNP..66..239A}. Plotted in figures \ref{fig:hc}, \ref{fig:S} and \ref{fig:omega} is a box with an amplitude equal to an event releasing $10^{-4}\Msun c^{2}$ of energy, centred at a frequency of $100\,\textrm{Hz}$, over a period of $10\,\textrm{ms}$ and at a distance of $10\,\textrm{Mpc}$.

\subsubsection{Supernovae}
Simulations of GWs from a core collapse supernova event produce radiation of approximately $(10^{2}-10^{3})\,\textrm{Hz}$, \cite{2002A&A...393..523D}. The GW signal undergoes ${\cal{O}}(1)$ oscillation and hence is burst-like. \cite{2002A&A...393..523D} calculate the average maximum amplitude of GWs for a supernova at distance $r$ as
\begin{equation} h_{0}=8.9\times 10^{-21}\left( \frac{10 \,\textrm{kpc}}{r} \right) \quad .\end{equation}
Adopting an expected event rate for supernova of $\rho_{\textrm{SN}}=5\times10^{-4}\textrm{Mpc}^{-3}\textrm{yr}^{-1}$ (\cite{2013ASPC..467...59S}) a distance is chosen such that we expect one supernova per 10 years, $r\approx 3 \,\textrm{Mpc}$, this gives an amplitude of $h_{c}=10^{-22}$.




\subsection{Sources for space based detectors}

\subsubsection{Massive black hole binaries}
Space based detectors will be sensitive to equal mass mergers in the range $(10^{4}-10^{7})\,\Msun$, \cite{JohnsLivingReview}. Predictions of the event rate for these megers range from ${\cal{O}}(1-100)\,\textrm{yr}^{-1}$ for LISA with signal-to-noise ratios of up to 1000. The uncertainty in this rate reflects our uncertainty in the growth mechanisms of the supermassive black hole population. Plotted in figures \ref{fig:hc}, \ref{fig:S} and \ref{fig:omega} is a box with signal-to-noise ratio of 100 for eLISA at it's peak sensitivity. The range of frequencies plotted is $(3\times 10^{-4}-3\times 10^{-1})\textrm{Hz}$, this corresponds to circular binaries in the mass range quoted above.

\subsubsection{Galactic binaries} \label{sec:GB}
These divide into two classes, the unresolvable and the resolvable galactic binaries. The distinction between resolvable and unresolvable is detector specific; here we choose LISA. This boundary will not be too different for eLISA but would move substantially for either of the decihertz detectors. For the unresolvable binaries the box plotted corresponds to the estimate of the background due to \cite{Nelemans},
\begin{equation} h_{c}(f)= 5\times 10^{-21} \left(\frac{f}{10^{-3}\,\textrm{Hz}}\right)^{-2/3} \quad . \end{equation}
For the resolvable binaries the event rates range from ${\cal{O}}(10^{4})$ for LISA to ${\cal{O}}(10^{3})$ for eLISA, \cite{JohnsLivingReview}. Signal-to-noise ratios for these events observed with eLISA will extend to above 50. The box plotted in figures \ref{fig:hc}, \ref{fig:S} and \ref{fig:omega} has a signal-to-noise ratio of 50 for eLISA at its peak sensitivity, the frequency range of the box is based on visual inspection of Monte Carlo population simulation results presented in \cite{DoingScienceWitheLISA}.

\subsubsection{Extreme mass ratio inspirals}
EMRI events occur when a compact stellar mass object inspirals into a supermassive black hole. There is extreme uncertainty in the event rate for EMRIs due to the poorly constrained astrophysics in galactic centres; event rate estimates per galaxy range from ${\cal{O}}(1-10^{3})\,\textrm{Gyr}^{-1}$ for black hole EMRIs and ${\cal{O}}(10-5000)\,\textrm{Gyr}^{-1}$ for white dwarf and neutron star EMRIs. The box plotted in figures \ref{fig:hc}, \ref{fig:S} and \ref{fig:omega} has a characteristic strain of $h_{c}=3\times 10^{-20}$ at $10^{-2}\;\textrm{Hz}$ which corresponds to a $10\Msun$ BH inspiralling into a $10^{6}\Msun$ supermassive black hole with a last stable orbital eccentricity of $e=0.3$ at a distance of $1\,\textrm{Gpc}$. The frequency width of the box is somewhat unknown, EMRI events can occur into a black hole of any mass, and hence EMRIs can occur at any frequency. However there is substantial uncertainty in the black hole mass function.




\subsection{Sources for pulsar timing arrays}
There are several interesting potential sources of GWs in this frequency band not discussed further here; for example cosmic strings and the reheating process.

\subsubsection{Supermassive black hole binaries}
The current best published limit for the amplitude of the stochastic background is $h_{c}=6\times 10^{-15}$ at frequency of $f_{0}=1\,\textrm{yr}^{-1}$, \cite{Haasteren}. There is strong theoretical evidence that the actual background lies close to the current limit, \cite{imminentdetectionofgravitationalwaves} and \cite{NONimminentdetectionofgravitationalwaves}. As the frequency increases the sources become less redshifted and hence louder until, at a certain frequency, they will cease to be a background and become individually resolvable. The exact distinction between resolvable and unresolvable is detector specific. Plotted in figures \ref{fig:hc}, \ref{fig:S} and \ref{fig:omega} for the unresolvable background is a third of the limit due to \cite{Haasteren} with a cut off frequency of $f=1\,\textrm{yr}^{-1}$ which is suggested by Monte Carlo population studies, \cite{SesanaVecchioColancino}. For the resolvable sources the exact limit due to \cite{Haasteren} is plotted.

