\section{Introduction}
There are many ways of describing both the sensitivitiy of a gravitational wave (GW) detector and the strength of a GW source as a function of frequency. It is desirable to have a consistent convention between detectors and sources which is applicable across all frequencies and which allows information about both detectors and sources to be plotted on the same graph. Another desirable feature of such a plot is that the detectors and sources are characterised in such a way that their relative heights gives a measure of the sources' detectability. 

In this work we tackle the differing conventions common in GW astronomy. When discussing the loudness of sources and the sensitivity of detectors there are three commonly used parameters based upon the strain: the characteristic strain, the power spectral density and the spectral energy density. We aim to disambiguate these three and give a concrete comparison of different detectors. It is hoped that this will provide a useful reference to the new and old alike. In the following two sections we present a summary of the various conventions and the relationships between them. A list of detectors (current and future) is given in section~\ref{sec:detectors} and a list of different GW sources is given in section~\ref{sec:sources}. In \ref{app:a} several example sensitivity curves are presented; these are also made available online.



\section{Characteristic amplitudes}
\subsection{Characteristic strain}

A source of GWs radiates in two polarisation states with amplitudes $h_{+}$ and $h_{\times}$; the sensitivity of our detector to each of these states will depend upon the relative orientations of the source and detector. The most obvious quantity related to the detectability of a GW is the average strain amplitude,
\begin{equation}\label{eq:h0} h_{0}=\sqrt{\left< h_{+}^{2}+h_{\times}^{2} \right>}=\frac{1}{2}\left( \textrm{ r.m.s. amplitude } \right) \quad.\end{equation}
The angled brackets define averaging over all directions and over a wave period (the factor of $1/2$ comes from time averaging). This average strain does \emph{not} meet the criteria described above: for an inspiralling source the instantaneous amplitude can be well below the noise level in the detector even when the source is still detectable. This apparent discrepancy is because the inspiral continues over many orbits which allows the signal-to-noise ratio, $\varrho$, to accumulate to a detectable level. To account for this the \emph{characteristic strain}, $h_{c}$, is defined as
\begin{equation}\label{eq:hc} 
\varrho_{\textrm{r.m.s.}}^{2}= \int_{0}^{\infty}\textrm{d}f\; \frac{4\left|\tilde{h_{0}}(f)\right|^{2}}{S_{n}(f)} =\int_{-\infty}^{\infty} \textrm{d}\left(\log f\right)\; \left(\frac{h_{c}(f)}{h_{n}(f)}\right)^{2} \quad,
\end{equation}
where $h_{n}$ is the detector's sky averaged r.m.s. noise in a bandwidth equal to $f$. This is related to the sky averaged \emph{one-sided} noise power spectral density by
\begin{equation}\label{eq:powerspectraldensity} S_{n}(f)=\frac{h_{n}(f)^{2}}{f} \quad . \end{equation}
An alternative convention is to use the \emph{two-sided} power spectral density, $S^{(2)}_{n}(f)=\frac{1}{2}S_{n}(f)$. To complete the picture it is necessary to relate the two strains, $h_{c}$ and $h_{0}$, in a manner which satisfies (\ref{eq:hc}). The means to do this will depend upon the source. In attempting to plot all sources on the same diagram it is necessary to have a consistent convention for $h_{c}$. The important relationship between $h_{c}$ and $h_{0}$ is derived for various types of source in section \ref{sec:voc}. It should be noted that the strain amplitudes $h_{0}$, $h_{c}$ and $h_{n}$ are all dimensionless, while $S_{n}$ has units of inverse frequency. 

One sensible choice of quantities to plot on a sensitivity curve is $h_{n}$ for the detector and $h_{c}$ for the source (see figure \ref{fig:hc}). Using this convention the area between the source and detector curves is related to the signal-to-noise ratio via (\ref{eq:hc}). This convention allows the reader to ``integrate by eye" for a given detector to see how detectable a given source is. An additional advantage of this convention is that the values on the strain axis for the detector curve correspond to the noise amplitude in the detector. One downside to plotting characteristic strain is that the values on the strain axis do not directly relate to the amplitude of the waves from the source.


\subsection{Power spectral density}\label{sec:psd}
Another common quantity to plot on sensitivity curves for both detectors and sources is the square root of the power spectral density (see figure \ref{fig:S}), which from (\ref{eq:powerspectraldensity}) is given by
\begin{equation}\label{eq:temp1} \sqrt{S_{n}(f)}=h_{n}(f)f^{-1/2} \quad .\end{equation}
This quantity has one advantage over characteristic strain: integrating $S_{n}$ (multiplied by the detector response function, which is typically of order unity) over all frequencies gives the mean square noise strain in the detector. Let $n(t)$ be the instantaneous noise strain, due to a background say, and it's Fourier transform $\tilde{n}(f)$. The Fourier transform of the measured noise in the detector is given by the noise times a frequency response function, $\tilde{N}(f)={\cal{R}}(f)\tilde{n}(f)$. The mean square noise amplitude in the detector is then given by
\begin{eqnarray} \label{eq:meansquare}
\left< N^{2} \right> &= \lim_{\tau \rightarrow \infty} \frac{1}{2\tau} \int_{-\tau}^{\tau}\textrm{d}t\; N(t)^{2} \nonumber \\
&=\lim_{\tau\rightarrow\infty}\frac{1}{\tau}\int_{0}^{\infty}\textrm{d}f\;\left|\tilde{N}(f)\right|^{2} \nonumber\\
&= \int_{0}^{\infty}\textrm{d}f\; S_{n}(f){\cal{R}}(f)^{2} \quad ,
\end{eqnarray}
using the Wiener-Kinchin theorem. Where the power spectral density is defined by
\begin{equation}\label{eq:ps} S_{n}(f)=\lim_{\tau\rightarrow\infty}\frac{1}{\tau}\left|\tilde{n}(f)\right|^{2} \quad . \end{equation}
Comparing (\ref{eq:ps}) with (\ref{eq:temp1}) gives the noise strain in terms of the Fourier transform.
\begin{equation} h_{n}(f)^{2}=\lim_{\tau\rightarrow\infty}\frac{1}{\tau}f\left| \tilde{n}(f) \right|^{2} \end{equation}
Equation (\ref{eq:meansquare}) is the reason $\sqrt{S_{n}}$ is sometimes called the strain noise, \cite{Phinney}. This is the most commonly used quantity on sensitivity curves. However in one important regard it is less appealing than characteristic strain: the height of the source above the detector curve is no longer is directly related to the signal-to-noise ratio.

\subsection{Energy density}
A third quantity which is sometimes used is the spectral energy density in GWs, $S_{\textrm{E}}$ (not to be confused with $S_{h}$ defined in section \ref{sec:psd}). The spectral energy density is the energy per unit volume, per unit frequency and is related to $S_{h}$ via
\begin{equation}\label{eq:spectralenergydensity} S_{\textrm{E}}(f)=\frac{\pi c^{2}}{4G} f^{2}S_{h}(f) \quad , \end{equation}
\cite{HellingsDowns}. It is usual to define the dimensionless quantity $\Omega_{\textrm{GW}}$ as the energy density per logarithmic frequency interval normalised to the critical density of the universe,
\begin{equation}\label{eq:omega} 
\Omega_{\textrm{GW}}(f)=\frac{fS_{\textrm{E}}(f)}{\rho_{c}c^{2}}=\frac{\pi}{4G\rho_{c}}f^{2}h_{c}(f)^{2}=\frac{\pi}{4G\rho_{c}}f^{3}S_{h}(f)  \quad .
\end{equation}
The critical density of the universe is $\rho_{c}=\frac{3H^{2}}{8\pi G}$, where $H$ is the Hubble constant, which is commonly parametrised as $H=h\times 100\, \textrm{km}\,\textrm{s}^{-1}\,\textrm{Mpc}^{-1}$ (the $h$ here has nothing to do with strain). The most common quatity related to energy density to be plotted on sensitivity curves is $\Omega_{\textrm{GW}}h^{2}$ (figure \ref{fig:omega}). This quantity has one aesthetic advantage over the others: it automatically accounts for the fact that there is less energy in low frequency waves of the same amplitude and does not place the sensitivity curves of puslar timing arrays much higher than most ground based detectors. However, unlike characteristic strain, the area between the source and detector curves is no longer simply related to the signal-to-noise ratio.


