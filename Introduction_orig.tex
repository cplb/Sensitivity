\section{Introduction}


There are many ways of describing both the sensitivities of a gravitational wave detector and the strength of a gravitational wave source as a function of frequency. It is desirable to have a consistent convention between detectors and sources which is applicable across all frequencies and which allows both detectors and sources to be plotted on the same graph. Another desirable feature of such a plot is that the detectors and sources are plotted in such a way that their relative heights gives a measure of the sources detectability. In this section a summary is attempted of all the conventions in use for this purpose and the relationships between them. A nearly exhaustive list of different gravitational wave sources is given in section~\ref{sec:sources}, and a list of detectors (past, present and future) is given in section~\ref{sec:detectors}. In \ref{sec:a} several plots of the sensitivity curves are curves are given, which are also made available online.

A source of gravitational waves radiates in two polarisation states with amplitudes $h_{+}$ and $h_{\times}$, the sensitivity of our detector to each of these states will depend upon the relative orientations of the source and detector. The most obvious quantity related to the detectability of a gravitational wave  is the average amplitude,
\begin{equation}\label{eq:h0} h_{0}=\sqrt{\left< h_{+}^{2}+h_{\times}^{2} \right>}=\frac{1}{2}\left< \textrm{ r.m.s. amplitude } \right> \quad.\end{equation}
The angled brackets define averaging over all directions and over a wave period (the factor of $1/2$ comes from the time averaging). This average strain amplitude clearly doesn't meet the criteria described above; for an inspiralling source the instantaneous amplitude can be well below the noise level in the detector even when the source is still detectable. This apparent discrepancy is because the source undergoes many oscillations at each frequency allowing the signal to noise to accumulate over time. This led \cite{FinnThorne} to define the characteristic strain, $h_{c}$, as,
\begin{equation}\label{eq:hc} \left(\frac{S}{N}\right)_{\textrm{r.m.s.}}^{2}=\int \frac{\textrm{d}f}{f}\; \left(\frac{h_{c}(f)}{h_{n}(f)}\right)^{2}=\int \textrm{d}\left(\log f\right)\; \left(\frac{h_{c}(f)}{h_{n}(f)}\right)^{2} \quad.\end{equation}
Where $h_{n}$ is the detectors sky averaged r.m.s. noise in a bandwidth equal to $f$ which is related to the noise one sided power spectral density\footnote{An alternative convention sometimes used when defining the power spectral density is the two sided function, this is related to the convention used here via $S^{(2)}_{n}(f)=\frac{1}{2}S_{n}(f)$.} (superscript S.A. denotes sky averaging).
\begin{equation}\label{eq:powerspectraldensity} h_{n}(f)=\sqrt{fS_{n}^{\textrm{S.A.}}(f)} \end{equation}
In order to complete the picture it is necessary to relate the two strains, $h_{c}$ and $h_{0}$ in a manner which satisfies (\ref{eq:hc}). The way to do this will depend upon the nature of the source. For an inspiralling binary the source spends a variable amount of time in each frequency band gradually accumulating signal to noise ratio, if $\phi$ is the orbital phase then the length of time spent at frequency $f$ is given by,
\begin{equation}\label{eq:inspiral}
 \frac{1}{2\pi}\frac{\textrm{d}\phi}{\textrm{d}\left(\log f\right)} = \frac{f}{2\pi}\frac{\textrm{d}\phi}{\textrm{d}f}=\frac{f^{2}}{\dot{f}} \quad .
\end{equation}
This leads to the definition of characteristic strain (see \cite{FinnThorne}) for inspirals in (\ref{eq:hc1}), where the factor of two is inserted to cancel the two arising in (\ref{eq:h0}).
\begin{equation}\label{eq:hc1} h_{c}(f)=h_{0}(f)\sqrt{\frac{2f^{2}}{\dot{f}}} \end{equation}
It should be noted that $h_{0}$, $h_{c}$ and $h_{n}$ are all dimensionless, because they are the strain amplitudes of the waves at a particular frequency, while $S_{n}$ has units of inverse frequency. Equation \ref{eq:hc1} is the relation between $h_{c}$ and $h_{0}$ for an inspiralling source, for other types of source an new definition which satisfies (\ref{eq:hc}) will have to found. Equation \ref{eq:hc} should be considered as the definition of characteristic strain and (\ref{eq:hc1}) a consequence of it for inspirals. In attempting to plot all sources on the same diagram it is necessary to have a consistent convention for $h_{c}$, so the relationship between $h_{c}$ and $h_{0}$ is very important, to emphasise this I give it a suitably pompous name, the vociferosity relation, (\ref{eq:hc1}) is the vociferosity relation for inspirals.

So one sensible choice of quantities to plot on a sensitivity curve would be $h_{n}$ for the detector and $h_{c}$ for the source, see figure \ref{fig:hc}. Using this convention, if the frequency is plotted on a logarithmic scale and the strain on a linear scale then the area between the source and detector line represents the signal to noise ratio. This convention allows the reader to ``Integrate by eye" for a given detector to see how detectable a given source is. An addition advantage of this convention is that the values on the strain axis for the detector curve correspond to the noise amplitude in the detector. The one downside to plotting characteristic strain is that the values on the strain axis do not directly relate to the amplitude of the waves from the source, they are related via the vociferosity relation.

Another common quantity to plot on sensitivity curves for both detectors and sources is the square root of the power spectral density (see figure \ref{fig:S}), which from (\ref{eq:powerspectraldensity}) is given by,
\begin{eqnarray}\label{eq:temp1} \sqrt{S^{\textrm{S.A.}}_{n}(f)}&=h_{n}(f)f^{-1/2} \quad\textrm{for detectors,} \nonumber \\
\sqrt{S_{h}(f)}&=h_{c}(f)f^{-1/2} \quad\textrm{for sources.}\end{eqnarray}
This quantity has one nice advantage over characteristic strain, integrating the quantity $S^{\textrm{S.A.}}_{n}$ (times a detector response function, which is often of order unity) over all frequencies gives the mean square noise strain in the detector. If the instantaneous noise strain in the detector is $n(t)$, due to a background say, then it's Fourier transform is defined via (\ref{eq:FT}),
\begin{equation}\label{eq:FT} \tilde{n}(f)=\int_{-\infty}^{\infty}\textrm{d}t\; n(t)\exp \left(-2\pi ift\right) \quad . \end{equation}
Since $n(t)$ is dimensionless, $\tilde{n}(f)$ has units of inverse frequency. The Fourier transform of the measured noise in the detector, $N(t)$, is given by the noise times a frequency response function,
\begin{equation} \tilde{N}(f)={\cal{R}}(f)\tilde{n}(f)\quad . \end{equation}
The mean square noise amplitude in the detector is then given by,
\begin{equation} \left< N(t)^{2} \right> = \lim_{\tau \rightarrow \infty} \frac{1}{2\tau} \int_{-\tau}^{\tau}\textrm{d}t\;n(t)^{2}=\lim_{\tau\rightarrow\infty}\frac{1}{\tau}\int_{0}^{\infty}\textrm{d}f\;\left|\tilde{n}(f)\right|^{2} \end{equation}
where the Wiener-Kinchin theorem has been used. It then follows that,
\begin{equation}\label{eq:meansquare} \left< N(t)^{2} \right> = \int_{0}^{\infty}\textrm{d}f\; S^{\textrm{S.A.}}_{n}(f){\cal{R}}(f)^{2} \end{equation}
where the power spectral density is given by,
\begin{equation}\label{eq:ps} S^{\textrm{S.A.}}_{n}(f)=\lim_{\tau\rightarrow\infty}\frac{1}{\tau}\left|\tilde{n}(f)\right|^{2} \quad . \end{equation}
It is straightforward to show that (\ref{eq:ps}) is consistent with the previous definition in (\ref{eq:powerspectraldensity}). Let $\tilde{n}(f_{i})=h_{n}(f)\delta (f_{i}-f)$ and substitute this into (\ref{eq:ps}),
\begin{equation} S^{\textrm{S.A.}}_{n}(f)=\lim_{\tau \rightarrow \infty}\frac{1}{\tau}\left| h_{n}(f)\delta (f_{i}-f) \right|^{2} \end{equation}
\begin{equation} S^{\textrm{S.A.}}_{n}(f)=\frac{h_{n}(f)^{2}}{f} \end{equation}
\begin{equation} h_{n}(f)=\sqrt{fS_{n}^{\textrm{S.A.}}(f)} \quad . \end{equation}
Equation \ref{eq:meansquare} is this reason $\sqrt{S^{\textrm{S.A.}}_{n}}$ is sometimes called the strain noise, \cite{Phinney}. This is probably the most commonly used quantity on sensitivity curves and has units of $\textrm{Hz}^{-1/2}$. However in several ways it is much less appealing than characteristic strain. Firstly the height of the source above the detector curve is no longer is directly related to the signal to noise ratio. Secondly, although the integral of $\sqrt{S_{n}}$ over frequency is a strain, we are plotting spectral quantities and $\sqrt{S_{n}}$ is not straightforwardly related to the strain in the detector.

A third quantity which is sometimes used is the energy density in gravitational waves. The quantity $S_{h}$ is not to be confused with the quantity $S_{\textrm{E}}$ defined by, for example, \cite{HellingsDowns}. This is the spectral energy density, which is the energy per unit volume, per unit frequency and is related to $S_{h}$ via,
\begin{equation}\label{eq:spectralenergydensity} S_{\textrm{E}}(f)=\frac{\pi c^{2}}{4G} f^{2}S_{h}(f) \quad . \end{equation}
So the total energy density of gravitational waves is given by the integral in (\ref{eq:epsilon}), where the quantity $\Omega_{\textrm{GW}}$ has been defined as the spectral energy density per unit logarithmic frequency interval normalised by the critical energy density of the universe to make it dimensionless\footnote{Just to add to the potential confusion there is another power spectrum quantity widely used in the pulsar timing community, usually denoted $P(f)$, which is the power spectra of timing residuals. This is related to the quantities described here via $P(f)=\frac{h_{c}(f)^{2}}{12\pi^{2}f^{3}}$, see \cite{Jenet}.}.
\begin{equation}\label{eq:omega} \Omega_{\textrm{GW}}(f)=\frac{fS_{\textrm{E}}(f)}{\rho_{c}c^{2}}=\frac{\pi}{4G\rho_{c}}f^{2}h_{c}(f)^{2}=\frac{\pi}{4G\rho_{c}}f^{3}S_{h}(f)  \end{equation}
\begin{equation}\label{eq:epsilon} \textrm{energy density}=\int \textrm{d}\left( \log f \right)\; \Omega_{\textrm{GW}}(f) \rho_{c}c^{2}\end{equation}
The critical density of the universe is $\rho_{c}=\frac{3H^{2}}{8\pi G}$, where $H$ is the Hubble constant, which is commonly parametrised as $H=h\times 100\, \textrm{km}\,\textrm{s}^{-1}\,\textrm{Mpc}^{-1}$ (the $h$ here has nothing to do with strain). So the most common quatity related to energy density to be plotted on sensitivity curves is $\Omega_{\textrm{GW}}h^{2}$, see figure \ref{fig:omega}. This quantity has one visual advantage over the previous two, it automatically accounts for the fact that there is less energy in low frequency waves of the same amplitude and doesn't place the sensitivity curves of puslar timing arrays much higher than most ground based detectors. However, unlike characteristic strain, the area between the source and detector curves is no longer simply related to the signal to noise ratio. Another downside is the scale of the y-axis is now not related in any simple way to the amplitude of the gravitational waves or the amplitude of the noise in the detector.

\subsection{Stochastic backgrounds from binaries}
Aside from inspirals another important detectable form of gravitational waves is a stochastic background due to a population of individually unresolvable binaries. A background is best described in terms of the energy density in gravitational waves. The population of sources will in general be at cosmological distances, in which case we need to distinguish the frequency in the source rest frame, $f_{r}$, from the measured frequency, $f$, via the redshift, $f_{r}=(1+z)f$. The comoving number density of sources producing the background will also be a function of redshift, $n(z)$. If the sources producing the stochastic background are all in the local universe, as is the case for the background of unresolvable white dwarf galactic binaries, then simply set $n(z)=\delta (z)$ in all that follows. Equation \ref{eq:omega} gives an expression for the energy density per logarithmic frequency interval,
\begin{equation}\label{eq:stoch} fS_{\textrm{E}}(f)=\frac{\pi c^{2}}{4G}f^{2}h_{c}^{2} \quad . \end{equation}
Equation \ref{eq:stoch} relates the characteristic strain to the energy density which in turn depends on the gravitational wave amplitude, hence (\ref{eq:stoch}) is the vociferosity relation for a stochastic background. The fact that $h_{c}$ is given in terms of an energy density averaged over some region of space and not as a simple function of $h_{0}$ reflects the stochastic nature of the source. Let the total outgoing energy emitted in gravitation waves between frequencies $f_{r}$ and $f_{r}+\textrm{d}f_{r}$ by a single binary in our population be $\frac{\textrm{d}E_{\textrm{GW}}}{\textrm{d}f_{r}}\textrm{d}f_{r}$, then the energy density may be written as,
\begin{equation}\label{eq:Phinney} fS_{\textrm{E}}(f)=\int_{0}^{\infty}dz\; \frac{\textrm{d}n}{\textrm{d}z}\frac{1}{1+z}\frac{\textrm{d}E_{\textrm{GW}}}{\textrm{d}\left(\log f\right)} \quad . \end{equation}
For simplicity consider all the binaries comprising our background to be in circular orbits with frequencies $\nu=f_{r}/2$, and to be far from their last stable orbit so the quadrapole approximation holds. The chirp mass is defined as ${\cal{M}}=\mu^{3/5}M^{2/5}$, where $\mu$ is the reduced mass and $M$ the total mass. \cite{Thorne} gives the energy in gravitational waves from a single binary per logarithmic frequency interval,
\begin{equation} \frac{\textrm{d}E_{\textrm{GW}}}{\textrm{d}\left(\log f \right)}=\frac{G^{2/3}\pi^{2/3}}{3}{\cal{M}}^{5/3}f_{r}^{2/3} \quad . \end{equation}
An expression for characteristic strain can now be found (see, for example, \cite{SesanaVecchioColancino}).
\begin{equation}
h_{c}(f)^{2}=\frac{4G}{\pi c^{2}f^{2}}\int_{0}^{\infty}\textrm{d}z\;\int_{0}^{\infty}\textrm{d}{\cal{M}}\;\frac{\textrm{d}^{2}n}{\textrm{d}z\,\textrm{d}{\cal{M}}}\;\frac{1}{1+z}\frac{G^{2/3}\pi^{2/3}}{3}{\cal{M}}^{5/3}f_{r}^{2/3}
\end{equation}
\begin{equation}\label{eq:bigint}
h_{c}(f)^{2} =\frac{4G^{5/3}}{3\pi^{1/3}c^{2}}f^{-4/3}\int_{0}^{\infty}\textrm{d}z\;\int_{0}^{\infty}\textrm{d}{\cal{M}}\;\frac{\textrm{d}^{2}n}{\textrm{d}z\,\textrm{d}{\cal{M}}}\left( \frac{{\cal{M}}^{5}}{1+z} \right)^{1/3}
\end{equation}
From (\ref{eq:bigint}) it can be seen that the characteristic strain due to a stochastic background of binaries is a power law with index $\alpha=-2/3$. The amplitude of the background depends on the population statistics for the binaries under consideration and depends on the double integral in (\ref{eq:bigint}). The power law is usually parametrised as,
\begin{equation}\label{eq:power} h_{c}(f)= A\left(\frac{f}{\textrm{yr}^{-1}}\right)^{\alpha} \end{equation}
and constraints are then placed on $A$. In practice this power law will also have upper and lower frequency cut-offs related to the population of objects causing the spectra. The stochastic background due to other sources, such as cosmic strings or the reheating process, are also usually written in the same form as equation \ref{eq:power}, however they will have a different spectral indices, $\alpha$.

\subsection{Burst sources}
A signal is burst like if it's duration at a detectable amplitude is of order the wave period. If this is the case then the signal does not have time to accumulate signal to noise ratio in each frequency band in the way inspirals do. Hence the vociferosity relation is simply,
\begin{equation}\label{eq:simple} h_{c}(f)=h_{0}(f) \quad . \end{equation}


