\section{Types of source}\label{sec:voc}

\subsection{Inspirals}\label{sec:insp}
Inspiralling binaries spend a variable amount of time in each frequency band. If $\phi$ is the orbital phase then the number of cycles undergone at frequency $f$ is given by
\begin{equation}\label{eq:inspiral}
 \frac{1}{2\pi}\frac{\textrm{d}\phi}{\textrm{d}\left(\log f\right)} = \frac{f}{2\pi}\frac{\textrm{d}\phi}{\textrm{d}f}=\frac{f^{2}}{\dot{f}} \quad .
\end{equation}
This leads to a definition of characteristic strain for inspirals due to \cite{FinnThorne} (the factor of two arrises from a more careful treatment in \ref{app:b}),
\begin{equation}\label{eq:hc1} h_{c}(f)=h_{0}\sqrt{\frac{2f^{2}}{\dot{f}}} \quad . \end{equation}
Equation (\ref{eq:hc1}) is the relation between $h_{c}$ and $h_{0}$ for an inspiralling source; for other types of source a new definition which satisfies (\ref{eq:hc}) will have to be found. Equation (\ref{eq:hc}) should be considered as the definition of characteristic strain and (\ref{eq:hc1}) a consequence of it for inspirals. 




\subsection{Stochastic backgrounds}
Another important source of GWs is a stochastic background due to a population of individually unresolvable binaries. The population of sources will in general be at cosmological distances and it is necessary to distinguish the frequency in the source rest frame, $f_{r}$, from the measured frequency, $f$, via the redshift, $f_{r}=(1+z)f$. The comoving number density of sources, $n$, producing the background will also be a function of redshift; if the sources producing the stochastic background are all in the local universe then simply set $n(z)=\delta (z)$ in all that follows. Equation (\ref{eq:omega}) gives an expression for the energy density per logarithmic frequency interval,
\begin{equation}\label{eq:stoch} fS_{\textrm{E}}(f)=\frac{\pi c^{2}}{4G}f^{2}h_{c}(f)^{2} \quad . \end{equation}
The total energy emitted in GWs between frequencies $f_{r}$ and $f_{r}+\textrm{d}f_{r}$ by a single binary in the population is $\frac{\textrm{d}E_{\textrm{GW}}}{\textrm{d}f_{r}}\textrm{d}f_{r}$; the energy density may be written as
\begin{equation}\label{eq:Phinney} fS_{\textrm{E}}(f)=\int_{0}^{\infty}dz\; \frac{\textrm{d}n}{\textrm{d}z}\frac{1}{1+z}\frac{\textrm{d}E_{\textrm{GW}}}{\textrm{d}\left(\log f_{r} \right)} \quad , \end{equation}
where the factor of $\left( 1+z \right)^{-1}$ accounts for the redshifted energy.
For simplicity consider all the binaries comprising our background to be in circular orbits with frequencies $\nu=f_{r}/2$, and to be far from their last stable orbit so that the quadrapole approximation holds. The chirp mass is defined as ${\cal{M}}=\mu^{3/5}M^{2/5}$, where $\mu$ is the reduced mass and $M$ is the total mass of the binary. The number density of sources will also be a function of chirp mass, $n(z,{\cal{M}})$. \cite{Thorne} gives the energy in GWs from a single binary per logarithmic frequency interval as
\begin{equation}\label{eq:Thorne} \frac{\textrm{d}E_{\textrm{GW}}}{\textrm{d}\left(\log f \right)}=\frac{G^{2/3}\pi^{2/3}}{3}{\cal{M}}^{5/3}f_{r}^{2/3} \quad . \end{equation}
Using (\ref{eq:stoch}), (\ref{eq:Phinney}) and (\ref{eq:Thorne}) an expression for characteristic strain can now be found, \cite{SesanaVecchioColancino}.
\begin{equation}\label{eq:bigint}
h_{c}(f)^{2}=\frac{4G^{5/3}}{3\pi^{1/3}c^{2}}f^{-4/3}\int_{0}^{\infty}\textrm{d}z\;\int_{0}^{\infty}\textrm{d}{\cal{M}}\;\frac{\textrm{d}^{2}n}{\textrm{d}z\,\textrm{d}{\cal{M}}}\left( \frac{{\cal{M}}^{5}}{1+z} \right)^{1/3}
\end{equation}
From (\ref{eq:bigint}) it can be seen that the characteristic strain due to a stochastic background of binaries is a power law in frequency with spectral index $\alpha=-2/3$. The amplitude of the background depends on the population statistics of the binaries under consideration via $n(z,{\cal{M}})$. The power law is usually parametrised as
\begin{equation}\label{eq:power} h_{c}(f)= A\left(\frac{f}{f_{0}}\right)^{\alpha} \quad , \end{equation}
and constraints are then placed on $A$. In practice this power law will also have upper and lower frequency cut-offs related to the population of objects causing the spectra. The stochastic background due to other sources, such as cosmic strings or the reheating process, are also usually written in the same form as (\ref{eq:power}), however they will have a different spectral indices.




\subsection{Burst sources}
A signal is burst-like if it's duration at a detectable amplitude is of the same order as the wave period. If this is the case then the signal does not have time to accumulate signal-to-noise in each frequency band in the way inspirals do. Hence the relation between $h_{c}$ and $h_{0}$ is simply
\begin{equation}\label{eq:simple} h_{c}(f)=h_{0}(f) \quad . \end{equation}

