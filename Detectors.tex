\section{Detectors}\label{sec:detectors}

\subsection{Ground based detectors}
The interferometer detectors listed in table \ref{table:t} are all sensitive to GWs in the frequency range ${\cal{O}}(10-10^{3})\,\textrm{Hz}$. They all have multiple narrow spikes in their sensitivity curves due, in part, to resonant frequencies in the suspension systems used to isolate the mirrors: these have been removed in figures \ref{fig:hc}, \ref{fig:S} and \ref{fig:omega} for clarity. The detectors fall broadly into three categories: 1\textsuperscript{st} generation detectors which have already begun operating, 2\textsuperscript{nd} generation detectors currently under construction and 3\textsuperscript{rd} generation detectors at the planning stage. 

\begin{table}[h!]
\caption{\label{table:t} For the aVIRGO sensitivity curve an interpolation to the data published on \url{https://wwwcascina.virgo.infn.it/advirgo/}(2013) was used. For KAGRA an interpolation to the data for version D of the detector published on \url{http://gwcenter.icrr.u-tokyo.ac.jp/en/researcher/parameter}(2013) was used. For the remaining detectors simple analytic fits to the sensitivity curves due to \cite{Sathyaprakash} were used.}
\begin{indented}
\item[]\begin{tabular}{ l l l l l }
\br
{\bf Detector} & {\bf Country} & {\bf Arm length} & {\bf  Approximate date} & {\bf Generation} \\
\mr
  GEO600 	&	Germany 	& $600\,\textrm{m}$ 	& 2001-present 	   & 1$^{\textrm{st}}$\\
  TAMA300 	& 	Japan		& $300\,\textrm{m}$ 	& 1995-present     & 1$^{\textrm{st}}$\\
  iLIGO		&	US		& $4\,\textrm{km}$ 	& 2004-2010 	   & 1$^{\textrm{st}}$\\
  iVIRGO	& 	Italy		& $3\,\textrm{km}$ 	& 2007-2010 	   & 1$^{\textrm{st}}$\\
  aLIGO 	&	US		& $4\,\textrm{km}$ 	& \emph{est.} 2016 & 2$^{\textrm{nd}}$\\
  KAGRA		&	Japan		& $3\,\textrm{km}$ 	& \emph{est.} 2018 & 2$^{\textrm{nd}}$\\
  aVIRGO	&	Italy	 	& $3\,\textrm{km}$ 	& \emph{est.} 2017 & 2$^{\textrm{nd}}$\\
  ET		&	Italy		& $10\,\textrm{km}$ 	& \emph{est.} 2025 & 3$^{\textrm{rd}}$\\
\br
\end{tabular}
\end{indented}
\end{table}




\subsection{Space based detectors}
Space based detectors are sensitive to lower frequency gravitational waves than their ground based counterparts. This is partly because space based detectors can have much longer arms and partly because they are unaffected by seismic noise which limits the low frequency performance of ground based detectors. The classic space based detector is LISA, all the sources discussed here lie within LISA's sensitivity curve. Here we divide the proposed missions into two classes: the milli-Hz detectors LISA and eLISA, and the deci-Hz detectors DECIGO and BBO.

\subsubsection{LISA and eLISA}
We use an analytic fit to the instrumental noise curve given by \cite{Sathyaprakash}. When observing individual sources with LISA there is an additional contribution to the noise from a background of unresolvable binaries. This is not included here as we consider the background as a source of GWs, see section \ref{sec:GB}. eLISA is a re-scoped version of the classic LISA mission with slightly reduced peak sensitivity shifted to higher frequencies. We use an analytic fit to the instrumental noise curve given by \cite{DoingScienceWitheLISA}.
\subsubsection{DECIGO and BBO}
These missions are designed to probe the decihertz region of the GW spectrum, both are considerably more ambitious than the LISA or eLISA mission and are likely to be launched further into the future. Here simple analytic fits to the sensitivity curves given by \cite{2011PhRvD..83d4011Y} are used.





\subsection{Pulsar timing arrays}
PTAs can be thought of as naturally occuring interferometers with galactic scale arm lengths, hence they are sensitive to much lower frequencies than the detectors considered so far.
Each pulsar is a very regular clock and the measured arrival time can be compared against a prediction leaving a residual which includes the effects of passing GWs. \cite{SesanaVecchioColancino} describe the shape of the sensitivity curve obtained by correlating the timing residuals from each of the $N_{p}$ pulsars in the array. Let the total timing residual be the sum of the residuals due to noise and due to GWs,
\begin{equation} \delta t= \delta t_{n} + \delta t _{h} \quad .\end{equation}
An upper limit to any background may be placed by assuming the residuals are due entirely to GWs. First consider the case of two pulsars at distance $d$ and with $\delta t_{h}\approx h_{0}/f$. In this case the smallest detectable signal is
\begin{equation} h^{2}\Omega_{\textrm{GW}}(f) \propto \frac{\delta t^{2}_{\textrm{r.m.s.}}f^{4}}{\sqrt{T\Delta f}} \quad , \end{equation}
where $T$ is the observation time and $\Delta f$ the bandwidth of the search. When using more than two pulsars each separate pair of pulsars forms an independent detector, so the number of detectors scales as $N_{p}^{2}$. The optimum signal-to-noise ratio is given by adding the individual signal-to-noise ratios in quadrature. Hence the total sensitivity is given by
\begin{equation} h^{2}\Omega_{\textrm{GW}}(f) \propto \frac{\delta t^{2}_{\textrm{r.m.s.}}f^{4}}{N_{p}\sqrt{T\Delta f}} \quad . \end{equation}
This can be related to the characteristic strain using (\ref{eq:omega}),
\begin{equation}\label{eq:PTA} h_{c}(f) \propto \frac{\delta t_{\textrm{r.m.s.}}f}{N_{p}^{1/2}\left( T\Delta f \right)^{1/4}} \quad . \end{equation}
The characteristic strain that the PTA is sensitive to scales linearly with $f$ between a mimumum frequency of approximately $T^{-1}$ and a maximum frequency $\Delta T ^{-1}$, where $\Delta T$ is the gap between pulsar observations. This gives the wedge shaped curves plotted in figures \ref{fig:hc}, \ref{fig:S} and \ref{fig:omega}. The absolute values of the sensitivity is fixed by normalising (\ref{eq:PTA}) to agree with a limit at a given frequency for each PTA.

There is a discrepancy between the treatment of PTA sensitivity curves here and the sources for the higher frequency detectors discussed in section \ref{sec:sources}. When observing a long lived source with a high frequency detector the convention was to define a new \emph{characteristic} strain to satisfy (\ref{eq:hc}). Here the convention is to leave the strain untouched and instead move the PTA sensitivity curve to account for the length of observation time. This discrepancy is an unfortunate result of the conventions in use by the different GW communities.


\subsubsection{EPTA/PPTA/NANOGrav}
The PTAs currently in operation are the EPTA, PPTA and NANOGrav. There are published limits on the amplitude of the stochastic background from all three detectors: currently the lowest is from the EPTA, \cite{Haasteren}. The EPTA curves in the figures show the  published limit based on an analysis of 7 pulsars over approximately 10 years.

\subsubsection{IPTA}
Combining the existing arrays would yield a single PTS using approximately 30 pulsars. The curves plotted in the figures are based on 3 times as many pulsars as the EPTA curves and timed for twice as long.

\subsubsection{SKA}
Following \cite{SesanaVecchioColancino} it is assumed that the SKA is able to measure $\delta t^{2}_{\textrm{r.m.s.}}$ a factor of 10 better than the current PTAs, the curves drawn in the figures are based on 3 times as many pulsars timed for 5 times as long as the current EPTA curves.





