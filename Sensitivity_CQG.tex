\documentclass[12pt]{iopart}
\usepackage{iopams}  

\newcommand{\Msun}{\ensuremath{M_{ \odot }}}

\begin{document}

\title[GW sensitivity curves]{The full spectrum of gravitational wave sensitivity curves}

\author{C J Moore, C P L Berry, R H Cole and J R Gair}

\address{Institute of Astronomy, Madingley Road, Cambridge, CB3 0DS, United Kingdom}
\ead{cjm96@cam.ac.uk}

\begin{abstract}
When discussing the sensitivity of gravitational wave detectors there are several common conventions. These are frequently confused. We outline the merits of and differences between the various quantities used for parameterizing noise curves and characterizing gravitational wave amplitudes. We conclude by producing plots that consistently compare different detectors.
\end{abstract}

\pacs{04.30.--w, 04.30.Db, 04.80.Nn, 95.55.Ym}
\submitto{\CQG}
\maketitle

\section{Introduction}

``Differing weights and differing measures---the {\sc Lord} detests them both.'' (Proverbs 20:10)\\

\noindent{}Even the best of us can be frustrated by the use of different conventions or units when describing a single quantity. It is a common occurrence in physics that  there is a particular parameterization that is perfect for one experiment or calculation, but this is not of universal applicability. When comparing results across studies it is then necessary to trawl through the literature to check that $x$ is really the same parameter. Astronomy and astrophysics are rife with specialized units, from the jansky (a spectral flux density of $10^{-26}$ watts per square metre per hertz) to the solar neutrino unit (a neutrino flux that produces an interaction rate of $10^{-36}$ per second per target atom). Inevitably, things get lost in translation. The mismatch of conventions can be confusing to those new to the field, especially students.

In this work we tackle the differing conventions common in gravitational wave astronomy. The amplitude of a gravitational wave is a strain, a dimensionless quantity. When discussing the loudness of sources and the sensitivity of detectors there are three commonly used parameters based upon the starin: the characteristic strain, the power spectral density and the energy spectrum. We aim to disambiguate these three and give a concrete comparison of different detectors. It is hoped that this will provide a useful reference to the new and old alike. 

\ack
CJM, CPLB and RHC are supported by STFC. JRG is supported by the Royal Society.

\section*{References}
%\begin{thebibliography}{10}
%\end{thebibliography}

\end{document}

